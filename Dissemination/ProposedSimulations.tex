\documentclass[12pt]{article}
\usepackage{amsmath,amsfonts}
\usepackage{epsfig}
\usepackage{lscape}
%\usepackage{rotating}
\usepackage{array}
\usepackage{natbib}
\usepackage{graphicx}
\usepackage{enumerate}
\topmargin 0in \headheight 0.0in \textheight 9in \textwidth 6.5in
\oddsidemargin 0.1in \evensidemargin 0.1in
\renewcommand{\baselinestretch}{1}
\newcommand{\beq}{\begin{equation}}
\newcommand{\eeq}{\end{equation}}
\newcommand{\ie}{\emph{i.e.}}
\newcommand{\eg}{\emph{e.g.}}



\graphicspath{{../Figures/}}






\begin{document}

\section*{Simulated data structures}

This section outlines in a unified fashion how we will simulate data without the specifications of each simulation context.

We will simulated $n=n_A+n_B$ images, where $n_A$ is the number of images from group $A$ and $n_B$ is the number of images from group $B$. Let $y_i$ be the $i$th image structured as an $s\times 1$ vector corresponding with the $s$ pixels/voxels of the image. Let $z_i$ be 1 if image $i$ is in group $A$ and 0 if image $i$ is in group $B$. We specify two covariance matrices $\Sigma_g=Q_g\Lambda^g Q_g^T$ for $g=A,B$ with matrix of eigenvectors $Q_g$ and eigenvalues $\{\Lambda_{ii}^g\}_{i=1}^s$, and a group effect $\beta\in\mathbb R$. Then we simulate images independently from the model

$$y_i =\beta z_i + \epsilon_i\hspace{2cm}\epsilon_i\sim\text{MVN}(0,\Sigma_{g_i}),$$
equivalently
$$y_i\sim\text{MVN}(\beta z_i,\Sigma).$$

A particular simulation context will be defined through specification of the following:
\begin{enumerate}[(a)]
	\item An `image space.' Unless otherwise stated, we'll be talking about images of the same dimension as the handwritten digit data. Thus, $s=256$ pixels arranged in a $16\times16$ lattice.
	\item Desired number of observations per group, $n_A$ and $n_B$
	\item Desired covariances $\Sigma_g$, either by providing a model (e.g. exponential) or by providing eigenvectors $Q_g$ and eigenvalues $\Lambda_g$. Unless otherwise specified, we will assume $\Sigma_A=\Sigma_B$.
	\item The group effect $\beta$. We might choose $\beta$ in a couple of ways, such as by a function of pixel location (e.g. 1 for all pixels in the top half of the image and -1 for all pixels in the bottom half), as a linear combination of ESF/GSP eigenvectors, or empirically (using the empirical covariance matrix, maybe estimated with regularization).
\end{enumerate}


\section*{Models}

\begin{enumerate}[(a)]
	\item Predicting images $y_i$ from group $z_i$/Inferring group effect
		\begin{enumerate}[(1)]
			\item VBM
			\item spVBM
			\begin{enumerate}
				\item Only positive eigenvalue eigenvectors?
				\item Only a subset of eigenvectors?
				\item Knots or exact computation?
			\end{enumerate}
		\end{enumerate}
	\item Predicting group $z_i$ from image $\tilde y_i$ using sparse logistic regression where $\tilde y_i$ is a transformation of the image $y_i$
	\begin{enumerate}[(1)]
		\item Voxels as covariates $\tilde y_i=y_i$
		\item functional PCs as covariates
		\item Frequency intensities as covariates (images after application of ESF or GSP transformation, e.g. $\tilde y_i=Q_{ESF}y_i$)
		\item One other method?
	\end{enumerate}
	\item Inferring network
		\begin{enumerate}
			\item Do we have a specific methodology picked out to look at?
		\end{enumerate}
\end{enumerate}



\section*{Specific simulations of interest}

\begin{enumerate}
	\item First simulation
	\begin{enumerate}
		\item[Data] Let $n_A=n_B=1000$ where $\beta$ is 1 for pixels in the center $8\times8$ pixel square and 0 elsewhere, $\Sigma=\Sigma_A=\Sigma_B$ is an exponential correlation matrix with rate 1. 
		\item[PredictImage] Fit VBM and spVBM models predicting images $y_i$ from group $z_i$ using all eigenvectors from 16 knots. Report the same performance metrics as in Sarah's paper. Use an exponential network.
		\item[PredictGroup] Train sparse logistic regression models on 800 observations (400 per group) predicting group $z_i$ from image covariates $\tilde y_i$ for the following transformations:
		\begin{enumerate}
			\item Voxels as covariates (no transformation)
			\item ESF frequencies as covariates (transformation from exponential network)
			\item GSP frequencies as covariates (transformation from exponential network, unnormalized Laplacian)
			\item functional PCs as covariates (talk with Yue for details)
		\end{enumerate}
		Report test AUC, sensitivity, and specificity using remaining 200 observations. Report also which covariates were selected in the voxel, ESF frequency, and GSP frequency models.
	\end{enumerate}
\end{enumerate}



\end{document}

